\title{Homework 4 - Group 34}

\author{
        Sean Caster,
        Joshua Sean Bell,
        Tarren Engberg
}
\date{June 8th, 2018}

\documentclass[titlepage,draftclsnofoot,onecolumn]{article}
\usepackage{geometry}
\geometry{letterpaper, margin=0.75in}

\begin{document}
\maketitle

\begin{abstract}
  Group 34 describes their experience implementing a SLOB best fit algorithm for memory management in the Linux Yocto environment as well as the experience of making a custom system call to gather SLOB information for analysis of the difference between the default first-fit and the custom best-fit algorithms.
\end{abstract}

\section*{Write Up}
Group 34 commands used, flag definitions and question answers.

\paragraph{Design}
To complete this assignement we have decided to first figure out how we want to transform the first-fit algorithm to be a best-fit algorithm. The first-fit algorithm simply looks for the first slot it comes across in the page that is big enough to fit and places the data. The best fit method differs in that it searches the entire page for a large enough slot and also for the smallest of those slots that will fit the data. This best-fit method is designed to pack more densly, lowering the frequency of fragmentation. Once we have developed a modification to first-fit that makes it work as best-fit, we will build in a custom system call that allows us to record an enumeration of the fragmentation in memory so that the two algorithms may be compared for data density.

\paragraph{Version Control Log}
\texttt{
\begin{tabular}{|r|l|}
  \hline
  Commit: & 1467e54235ae27369886bda04c287d5028502211 \\ \cline{2-2}
  Author: & Tarren Engberg <engbergt@os2.engr.oregonstate.edu> \\ \cline{2-2}
  Date: & Sun May 27 22:53:58 2018 -0700 \\ \cline{2-2}
  Message: & Assignment 3 commit \\
  \hline
\end{tabular}
}

\paragraph{Work Log}
- We started by first examining the slob.c file in the mm directory in the Linux kernal. We found the function slob_alloc and began tracing it through to try and understand how it is working so that we may make modifications.

\paragraph{Main Point of the Assignment}
We believe the main purpose of this assignment is to first get familiarity with Linux module compilation. Also, this assignment gives us exposure to the Linux Crypto API and gives us experience using the included functions.

\paragraph{Our Approach}
Our approach to this assignment is to basically piece the assignment together using chunks of code found elsewhere in Linux. We started with the sbd.c file and edited the transfer section to encrypt the input and decrypt the output using the supplied Linux Crypto API. Before we put the block device into a module, we first figured out how to make a module of our own work in the Kernel. Once we got that figured out we were able to move the device to the module and finish development from there.

\paragraph{Solution Correctness}


\paragraph{What We Learned}
There were a number of key things that we learned by doing this assignment. By actually working hands-on with the driver, we were able to get a better understanding of how memory allocation works. We also learned a lot about how modules work in the Linux Kernel. Another key concept we learned about is using the Linux Kernel's Crypto API.

\paragraph{TA Steps to Follow}


\begin{enumerate}
	\item \texttt{source envivronment-setup-i586-poky-linux.csh}
	\item \texttt{monster}
	\item In the VM: user name=root
	\item \texttt{scp [your remote ssh address]:[your filepath]/linux-yocto-3.19/drivers/block/sbd.ko ./}
	\item \texttt{./insmod}
	\item \texttt{./mkfs}
	\item \texttt{./mount}
	\item \texttt{cd /mnt/ourmount/}
	\item \texttt{vi hi}
	\item Write some data to hi in vi
	\item Save changes to hi and wait for a something like "enc'ing ctxt: ptxt was: " to print to console (this indicates encryption is working)
	\item \texttt{vi hi} again to see that it has been encrypted
	\item \texttt{less -f /dev/sbd0}
	\item \texttt{P 800} (takes you a bit below the offset to to see the decrypted data)
  \end{enumerate}
  
  NOTE: steps 10-14 are necessary since accessing 'hi' through vi will load cached memory (meaning our altered disk read which holds the decryption function won't get called); however, less must query our filesystem normally, meaning sbd\_request will be called, leading to an instantiation of sbd\_transfer, where our decryption function lives.



\end{document}
