\title{Homework 1 - Group 34}

\author{
        Sean Caster,
        Joshua Sean Bell,
        Tarren Engberg
}
\date{April 15th, 2018}

\documentclass[titlepage,draftclsnofoot,onecolumn]{article}
\usepackage{geometry}
\geometry{letterpaper, margin=0.75in}

\begin{document}
\maketitle

\begin{abstract}
  Group 34 presents the commands used in order to set up the Yocto version of Linux running in a qemu virtual machine on the OS2 OSU server.
\end{abstract}

\section*{Write Up}
Group 34 commands used, flag definitions and question answers.

\paragraph{Outline}

\begin{itemize}
  \item A log of commands used to perform the requested actions.
  \begin{itemize}
    \item (in linux-yocto...) \texttt{make menuconfig}
    \item (in menuconfig) \texttt{General setup->local version = yocto-standardg34hw1}
    \item saved output as \texttt{.config}
    \item (in linux-yocto...) \texttt{make -j4 all}
    \item \texttt{source envivronment-setup-i586-poky-linux.csh}
    \item \texttt{qemu-system-i386 -gdb tcp::5534 -S -nographic -kernel linux-yocto-3.19/arch/x86/boot/bzImage -drive file=core-image-lsb-sdk-qemux86.ext4,if=virtio -enable-kvm -net none -usb -localtime --no-reboot --append "root=/dev/vda rw console=ttyS0 debug".}
    \item (in another shell from linux-yocto-3.19) \texttt{gdb -tui vmlinux}
    \item (in gdb) \texttt{target remote :5534}
    \item \texttt{continue}
    \item \texttt{uname=root}
    \item \texttt{uname -r outputs yocto-standardg34hw1 and kernel-version is confirmed}

  \end{itemize}
  \item An explanation of each and every flag in the listed qemu command-line
  \begin{itemize}
    \item \texttt{-gdb tcp:5534} allows qemu to be used with a debugger on port 5534
    \item \texttt{-S} keeps the CPU from running at startup
    \item \texttt{-nographic} disables qemu's graphical output
    \item \texttt{-kernel} tells qemu to use the given kernel image
    \item \texttt{-drive} defines a new drive
    \item \texttt{-enable-kvm} starts qemu in kernel virtual machine
    \item \texttt{-net none} tells qemu that no network devices should be configured
    \item \texttt{-usb} enables the USB driver
    \item \texttt{-localtime} to start at the current local time
    \item \texttt{--no-reboot} tells qemu to exit rather than rebooting
    \item \texttt{-append} indicates the command line to use
  \end{itemize}
  \item The main purpose of this assignment was to get the Yocto version of Linux running in a Virtual Machine on the OS2 server.
  \item The way we ensured that our solution was correct was to run Yocto and view shell output to verify it was running.
  \item We learned about qemu and Yocto, neither of which any of our team had heard of before.
  \item Version control log (formatted as a table)
  \newline
  \texttt{
  \begin{tabular}{|r|l|}
    \hline
    Commit: & ffefc7486e0f207ddbee6335ff4d33ac756e735e \\ \cline{2-2}
    Author: & Tarren Engberg engbergt@os2.engr.oregonstate.edu \\ \cline{2-2}
    Date: & Sat Apr 14 13:51:27 2018 -0700 \\ \cline{2-2}
    Message: & first commit \\
    \hline
  \end{tabular}
  }
  \item Work Log
  \begin{itemize}
    \item We first logged into the OS2 server and pulled v3.19 of Yocto down to our group 34 folder.
    \item We then switched the tag to v3.19.2
    \item We ran the commands as listed above in the Command Log to get Yocto up and running in qemu.
    \item We realized that we hadn't initialized our own working repo, so we did that and made our first commit, after the kernal was up and running.
  \end{itemize}

\end{itemize}


\end{document}
