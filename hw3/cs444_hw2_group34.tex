\title{Homework 3 - Group 34}

\author{
        Sean Caster,
        Joshua Sean Bell,
        Tarren Engberg
}
\date{May 27th, 2018}

\documentclass[titlepage,draftclsnofoot,onecolumn]{article}
\usepackage{geometry}
\geometry{letterpaper, margin=0.75in}

\begin{document}
\maketitle

\begin{abstract}
  Group 34 describes their experience implementing a RAM disk driver which presents a chunk of memory as a read/write encrypted block device for the Yocto version of Linux running in a qemu virtual machine on the OS2 Oregon State University server.
\end{abstract}

\section*{Write Up}
Group 34 commands used, flag definitions and question answers.

\paragraph{Design}
To complete this assignement we have decided to first figure out how to get Linux to recognize and use a module of our choosing. We will start by copying a module file that already exists and using it, as is, to be a new module and figuring out how to get Linux to reccognize and use it. Once we are confident we can get Linux to reccognize and use a module of our creation, we will then build a module with the starting point of the SBD file and editing it to use Linux built in Crypto API.

\paragraph{Version Control Log}

\paragraph{Work Log}
- We are considering copying the virtio file and pasting some encryption functions in it to modify it to make it read and write encrypted data.
- We will use module_param(myArgVar, argVarType, Permissions) to send the key to the driver.
- Ram backed block device driver - we are looking at this file now as a possible starting point for the assignment.
- Our first step is to copy the Ram backed block device driver and save it as a module to try to get that working.
- Brd -> yocto/devices/block/brd - we are trying to get this
- We found a way to force module loading in the configuration. We are trying it to see if it works with our module.
- We found an online resource detailing how to build modules against a running kernal https://www.kernel.org/doc/Documentation/kbuild/modules.txt
- In order to use the Krypto API, we used the Linux documentation to search for uses of the krypto functions for example, crypto_cipher_encrypt_one. This gave us use cases for the functions needed to encrypt and decrypt the input and output of the driver.

\paragraph{Main Point of the Assignment}
We believe the main purpose of this assignment is to first get familiarity with Linux module compilation. Also, this assignment gives us exposure to the Linux Crypto API and gives us experience using the included functions.

\paragraph{Our Approach}
Our approach to this assignment is to basically piece the assignment together using chunks of code found elsewhere in Linux. We started with the sbd.c file and edited the transfer section to encrypt the input and decrypt the output using the supplied Linux Crypto API.

\paragraph{Solution Correctness}


\paragraph{What We Learned}


\paragraph{TA Steps to Follow}


\begin{enumerate}
  \item Step 1
  \item Step 2 ..
\end{enumerate}


\end{document}
